\section{Conclusión}

Con el uso de la metodología propuesta en este trabajo se creó un sistema para obtener un conjunto de datos, el cual sirve como base para realizar experimentos con diferentes metodologías ya sean estadísticas o de inteligencia artificial para determinar la velocidad de vehículos u otros objetos, con la capacidad de ser adaptando a las necesidades o integrando más componentes para obtener otros parámetros que se puedan considerar de importancia.

El sistema muestra que es relativamente sencillo obtener una gran cantidad de parámetros de un video implementando algoritmos existentes de flujo óptico y modelos ya entrenados de aprendizaje máquina. Al mismo tiempo se puede comparar una metodología simple para calcular la velocidad con otros métodos más sofisticados como los modelos implementados con la biblioteca Scikit o la simple red neuronal implementada con la biblioteca Tensor Flow.

Sin embargo, uno de los mayores retos para la implementación de este sistema ha sido la toma de muestras, ya que se tiene que encontrar un lugar ideal donde haya suficiente flujo vehicular para obtener la mayor cantidad de datos posible, a esto hay que sumarle que en caso de integrar nuevos componentes las muestras deben ser tomadas nuevamente.

Otro punto importante que se debe mencionar es el tiempo de procesamiento de cada muestra, como se presentó anteriormente en el peor de los casos puede tomar hasta un 33\% más tiempo del que dura el video original, además de la necesidad de que una persona sea quien limpie y valide las muestras, sumando más tiempo al procesamiento de cada video.

Los resultados obtenidos con el conjunto de datos generado ofrecen una idea de los diferentes métodos que podemos implementar y obtener resultados aceptables, pudiendo mejorar aumentando el tamaño del conjunto de datos, cambiando o mejorando el modelo implementado, puesto que en el mejor de los casos se tiene un error absoluto promedio de 0.205 M/S lo que equivale a 0.738 K/H siendo esta una mejor precisión a la del radar utilizado. Los métodos de aprendizaje profundo han demostrado en distintas áreas que son capaces de aprender patrones con gran precisión, pero en el caso de los experimentos realizados, el aprendizaje profundo no dispuso de tan buenos resultados, esto debido a la sencillez de la red neuronal utilizada.
