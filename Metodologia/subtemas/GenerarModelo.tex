\subsection{Generar Modelo}
\label{cap:GenerarModelo}

Para determinar la velocidad utilizando el conjunto de datos generado por el sistema partimos de lo más sencillo utilizando la formula física para el cálculo de la velocidad, Ecuación \ref{eq:Velocidad}.

\begin{equation}
    \label{eq:Velocidad}
    Velocidad = \frac{Distancia}{Tiempo}
\end{equation}

En este caso las muestras fueron tomadas por un Radar que solo nos proporciona la velocidad en Kilómetros o en Millas, así que necesitamos convertir este valor a Metros sobre segundo, utilizando la Ecuación \ref{eq:ConvertMSKH}.

\begin{equation}
    \label{eq:ConvertMSKH}
    \frac{M}{S} = \frac{18}{5} \times \frac{K}{H}
\end{equation}

La Tabla \ref{tab:CaracteristicasSistema} muestra que contamos con los 3 parámetros necesarios para estimar la velocidad, sin embargo, tenemos el valor de la distancia en pixeles, por lo que tenemos que encontrar el valor de la distancia estimada utilizando los valores con los que contamos, lo cual quedaría como muestra la Ecuación \ref{eq:DistanciaEstimada}.

\begin{equation}
    \label{eq:DistanciaEstimada}
    Distancia\:Estimada = Tiempo \times Velocidad
\end{equation}

Una vez que contamos con la distancia estimada, debemos encontrar la correlación entre esta y la distancia en pixeles, con lo cual tenemos la siguiente Ecuación \ref{eq:EcuacionB}.

\begin{equation}
    \label{eq:EcuacionB}
    B = \frac{Distancia \: Estimada}{Distancia \: Pixeles}
\end{equation}

A este valor lo llamamos simplemente B, el valor de B es quien nos ayudara a calcular una velocidad estimada con respecto a la distancia y el tiempo, lo cual quedaría como muestra la Ecuación \ref{eq:VelocidadB}.

\begin{equation}
    \label{eq:VelocidadB}
    Velocidad = \frac{Distancia \: Pixeles}{Tiempo} \times B
\end{equation}

Sin embargo, esta fórmula nos serviría solo para un dato en específico generado por el sistema por lo cual debemos encontrar un valor de B que modele la gran mayoría de nuestras muestras o tener un error lo más bajo posible, para esto calculamos el valor de B para todas las muestras y calculamos el promedio de B, Ecuación \ref{eq:PromedioB}.

\begin{equation}
    \label{eq:PromedioB}
    \overline{B} = \frac{\sum B}{n}
\end{equation}

Una vez que tenemos nuestra B promedio podemos sustituirla por B en la Ecuación \ref{eq:VelocidadB} lo cual quedaría como muestra la Ecuación \ref{eq:VelocidadBPromedio}.

\begin{equation}
    \label{eq:VelocidadBPromedio}
    Velocidad = \frac{Distancia \: Pixeles}{Tiempo} \times \overline{B}
\end{equation}



