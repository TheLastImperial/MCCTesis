\chapter{Metodología}
\label{cap:metodologia}

En este capítulo se describe el proceso de obtención de la velocidad de vehículos a partir de secuencias de imágenes, para realizar esta tarea se recopiló un conjunto de datos (\textit{dataset}) suficientes para trabajar con ella.  A partir del \textit{dataset}, el problema de obtención de la velocidad fue dividido en tres procesos principales (extracción de características, modelo predictivo y obtención de la velocidad). Para la generación del \textit{dataset} fue necesario ir físicamente a una vía terrestre para tomar las muestras utilizando un radar de velocidad y una cámara de video. El proceso de extracción de características parte de los vídeos o muestras tomados y después aplicar técnicas de aprendizaje profundo el proceso genera un archivo \enquote*{csv} con las características necesarias de los vehículos en cuestión. A partir del archivo se aplican técnicas estadísticas de correlación con la finalidad de relacionar la distancia en píxeles con una distancia en metros, para posteriormente en el proceso de obtención de la velocidad estimar la velocidad del vehículo. 


La metodología completa es simple como muestra la Figura \ref{fig:MetodologiaDF}, pero cada uno de sus procesos individuales son más complejos para conseguir una buena predicción.


\begin{figure}[H]

    \begin{tikzpicture}

        \node(a)[rectangle, draw=black, minimum size=1cm]
            {Obtención de conjunto de datos};
        \node(b)[rectangle, draw=black, minimum size=1cm]
            [right=of a]
            {Obtención de velocidad};

        \draw[->] (a) -- (b);

    \end{tikzpicture}
    \centering
    \caption{Diagrama de flujo metodología.}
    \label{fig:MetodologiaDF}

\end{figure}


\section{Obtención de conjunto de datos}


La obtención del conjunto de datos se divide en dos etapas, la primera etapa corresponde a la toma de muestras, la cual implica ir al lugar con flujo constante de vehículos para obtener la mayor cantidad de datos. La segunda etapa es la limpieza de las muestras con la cual se busca solo tomar en cuenta los vehículos a los que se logró tomar correctamente la velocidad.
La Figura \ref{fig:DFCreacionCD} muestra las dos etapas internas con las que cuenta la obtención de conjunto de datos.

\begin{figure}[H]
    \centering
    \includegraphics[width=0.5\textwidth]{Metodologia/imgs/ObtencionConjuntoDatos.png}
    \caption{Proceso de obtención de conjunto de datos.}
    \label{fig:DFCreacionCD}
\end{figure}


%%%%%%%%%%%%%%%%%%%%%%%%%%%%%%%%%%%%%%%%%%%%%%%%%%%%%%%%%%%%%%%%%%%%%%%%%%%%%%%%
%%%%%%%%%%%%%%%%%%%%%%%%%%%%%%%%%%%%%%%%%%%%%%%%%%%%%%%%%%%%%%%%%%%%%%%%%%%%%%%%
%%%%%%%%%%%%%%%%%%%%%%%%%%%%%%%%%%%%%%%%%%%%%%%%%%%%%%%%%%%%%%%%%%%%%%%%%%%%%%%%
%%%%%%%%%%%%%%%%%%%%%%%%%%%%%%%%%%%%%%%%%%%%%%%%%%%%%%%%%%%%%%%%%%%%%%%%%%%%%%%%
%%%%%%%%%%%%%%%%%%%%%%%%%%%%%%%%%%%%%%%%%%%%%%%%%%%%%%%%%%%%%%%%%%%%%%%%%%%%%%%%

\subsection{Toma de muestras}

Para la toma de muestras se requirió de dos dispositivos con los cuales se busca la extracción del conjunto de datos que posteriormente se utilizará como entrenamiento de los modelos predictivos. Uno de los dispositivos es el radar Bushnell (Figura \ref{fig:RadarVelocidad}) el cual tiene una precisión de +/- 1.6 kilómetros por hora.

\begin{figure}[H]
    \centering
    \includegraphics[width=0.4\textwidth]{Metodologia/imgs/bushnell.jpg}
    \caption{Radar de velocidad Bushnell(\cite{bushnell}).}
    \label{fig:RadarVelocidad}
\end{figure}

Mientras que el otro dispositivo es una cámara de video, la cual puede ser un dispositivo especializado para esta tarea o cualquier otro dispositivo capaz de grabar video. 
Las configuraciones mínimas pueden ser una resolución de 854 x 480 a 30 fotogramas por segundo. Sin embargo, hay que tomar en cuenta que al tener una resolución baja se puede perder calidad en la imagen y tomar un área menor a la deseada. Mientras que tener fotogramas tan bajos ocasionará perder vehículos que pasan a una velocidad más alta.
Por otro parte al aumentar la resolución y los fotogramas por segundo ocasiona que al sistema le tome más tiempo procesar el video. Para este caso se utilizó la cámara de Smartphones un Xiaomi Redmi Note 7 y un iPhone X configurados a 60 FPS y una resolución de Full HD (1920 x 1080 pixeles).

Existe un cuidado especial a la hora de posicionar la cámara, ya que no se quiere que los experimentos sean considerados como cálculos de velocidad en 2D, para esto la toma de muestras se realizó en un lugar donde el tráfico vehicular pase con cierto grado de inclinación, sin apuntar la cámara directamente al costado de donde pasan los vehículos, como muestra la Figura \ref{fig:LugarMuestrasDataset}.

\begin{figure}[H]
    \centering
    \includegraphics[width=0.8\textwidth]{Metodologia/imgs/LugarMuestras.jpg}
    \caption{Lugar donde se tomaron las muestras.}
    \label{fig:LugarMuestrasDataset}
\end{figure}

Cabe mencionar que por practicidad, el proceso de toma de muestras fue realizado, en un mismo lugar de la ciudad de Culiacán, Sinaloa.

Dado que la cámara y el radar de velocidad son dispositivos que no están sincronizados entre sí, fue necesario la implementación de un mecanismo para asociar la lectura del radar con la toma del video.

%%%%%%%%%%%%%%%%%%%%%%%%%%%%%%%%%%%%%%%%%%%%%%%%%%%%%%%%%%%%%%%%%%%%%%%%%%%%%%%%
%%%%%%%%%%%%%%%%%%%%%%%%%%%%%%%%%%%%%%%%%%%%%%%%%%%%%%%%%%%%%%%%%%%%%%%%%%%%%%%%
%%%%%%%%%%%%%%%%%%%%%%%%%%%%%%%%%%%%%%%%%%%%%%%%%%%%%%%%%%%%%%%%%%%%%%%%%%%%%%%%
%%%%%%%%%%%%%%%%%%%%%%%%%%%%%%%%%%%%%%%%%%%%%%%%%%%%%%%%%%%%%%%%%%%%%%%%%%%%%%%%
%%%%%%%%%%%%%%%%%%%%%%%%%%%%%%%%%%%%%%%%%%%%%%%%%%%%%%%%%%%%%%%%%%%%%%%%%%%%%%%%

\subsection{Limpieza de las muestras}

Una vez tomadas las muestras, fue necesario realizar una limpieza de los datos, estos son, los vehículos a los que se les detectó la velocidad utilizando el radar. Con la limpieza de los datos se busca eliminar los vehículos a los que no se les tomo la velocidad y dejando solo aquellos que si fueron considerados.

La toma de las muestras se realizá a partir de dos dispositivos que no están especializados para esta tarea (Cámara de video y radar de velocidad), la limpieza de las muestras ayuda a combinar la información de ambos dispositivos realizando una inspección visual de los videos, en la cual se identifica el segundo del video en el que pasa el vehículo de interes, la velocidad detectada por el radar, el carril por el cual viaja el vehículo y una descripción del vehículo para futuras referencias. Estos cuatro datos son guardados en un archivo de texto separado por comas (csv) el cual servirá de entrada para el siguiente paso en la metodología.

Determinar la velocidad es una tarea que necesita el tiempo y la distancia que toma un vehículo en pasar de un lugar a otro. Para esto el sistema necesita ser configurado con un punto de entrada y un punto de salida en el eje X. A partir de ahora a estos dos puntos se les llamará simplemente por punto A y punto B.  Los puntos A y B deben ser colocados de manera que los vehículos deben pasar completamente por cada uno de ellos. Además, cada una de las muestras deben tener diferentes valores para los puntos A y B, sin embargo, para este caso no fue necesario implementar diferentes valores, ya que todos los videos fueron tomados en el mismo lugar. La Figura \ref{fig:LugarLimites} muestra donde fueron colocados los puntos A y B en las muestras, denotados por dos lineas azules verticales.

\begin{figure}[H]
    \centering
    \includegraphics[width=0.8\textwidth]{Metodologia/imgs/LugarLimites_01.jpg}
    \caption{Límites para lugar de las muestras.}
    \label{fig:LugarLimites}
\end{figure}

Una vez que se deciden los valores para los puntos A y B, se genera el archivo csv. Para esto se toma en cuenta el segundo en que pasa debe ser lo más cercano posible al centroide del vehículo cuando pasa por el punto B. Es importante ingresar una descripción, aunque esta no va a ser usada por el sistema. Se volverá importante para validar que el vehículo al que se le tomó la velocidad es el mismo que detecto el sistema.

%%%%%%%%%%%%%%%%%%%%%%%%%%%%%%%%%%%%%%%%%%%%%%%%%%%%%%%%%%%%%%%%%%%%%%%%%%%%%%%%
%%%%%%%%%%%%%%%%%%%%%%%%%%%%%%%%%%%%%%%%%%%%%%%%%%%%%%%%%%%%%%%%%%%%%%%%%%%%%%%%
%%%%%%%%%%%%%%%%%%%%%%%%%%%%%%%%%%%%%%%%%%%%%%%%%%%%%%%%%%%%%%%%%%%%%%%%%%%%%%%%
%%%%%%%%%%%%%%%%%%%%%%%%%%%%%%%%%%%%%%%%%%%%%%%%%%%%%%%%%%%%%%%%%%%%%%%%%%%%%%%%
%%%%%%%%%%%%%%%%%%%%%%%%%%%%%%%%%%%%%%%%%%%%%%%%%%%%%%%%%%%%%%%%%%%%%%%%%%%%%%%%

\section{Obtención de velocidad}

Como se mencionó anteriormente en la Sección \ref{cap:metodologia} la obtención de la velocidad se divide en tres partes las cuales están representadas en el Figura \ref{fig:DFObtencionDeVelocidad} y se describen a lo largo de esta sección.

\begin{figure}[H]
    \centering
    \begin{tikzpicture}

        \node(a0)[rectangle,
            draw=black,
            minimum width=7cm,
            minimum height=5.7cm,
            label=Obtención de velocidad]
            {};

        \node(a)[rectangle, draw=black, minimum width=5cm, minimum height=1cm] at ([yshift=-2em]a0.north)
        % [inside=of a0]
            {Extracción  de caracteristicas};

        \node(b)[rectangle, draw=black, minimum width=5cm, minimum height=1cm]
            [below=of a]
            {Modelo predictivo};

        \node(c)[rectangle, draw=black, minimum width=5cm, minimum height=1cm]
            [below=of b]
            {Obtención de la velocidad};

        \draw[->] (a) -- (b);
        \draw[->] (b) -- (c);

    \end{tikzpicture}

    \caption{Proceso de obtención de velocidad.}
    \label{fig:DFObtencionDeVelocidad}
\end{figure}

\subsection{Extracción de caracteristicas }

Una vez que se tienen las muestras y se genera el archivo CSV relacionado podemos ejecutar el sistema para extraer las características que nos interesa junto a la velocidad detectada por el radar de velocidad.

El sistema se encarga de leer el video utilizando la biblioteca OpenCV con la cual examina fotograma por fotograma, e identifica los vehículos utilizando la red neuronal YOLO, una vez que tiene identificados todos los vehículos dibuja la caja correspondiente a cada uno de ellos. La Figura \ref{fig:LugarDeteccion} muestra dos vehículos detectados, los cuales están dentro de un recuadro.

\begin{figure}[H]
    \centering
    \includegraphics[width=0.8\textwidth]{Metodologia/imgs/Deteccion.jpg}
    \caption{Detección de vehículos dentro de recuadros.}
    \label{fig:LugarDeteccion}
\end{figure}

El seguimiento de los vehículos se realiza por medio del Filtro Kalman para determinar su ubicación en el próximo fotograma. El sistema se encarga de guardar todas las ubicaciones de los vehículos en el transcurso del tiempo, con lo cual es capaz dibujar todo el trayecto que han tenido cada uno de ellos y con el uso de regresión lineal se crea una recta que corresponde a las trayectorias de los vehículos, la Figura \ref{fig:LugarSeguimiento} muestra un vehículo detectado en un recuadro blanco, así como un par de líneas una amarilla y otra roja que corresponden al seguimiento y a la recta calculada correspondiente al seguimiento.

\begin{figure}[H]
    \centering
    \includegraphics[width=0.8\textwidth]{Metodologia/imgs/Seguimiento.jpg}
    \caption{Detección de vehículos y sus trayectorias.}
    \label{fig:LugarSeguimiento}
\end{figure}

Otra característica que se identifica en la Figura \ref{fig:LugarSeguimiento} es la creación de un triángulo en color negro, con el cual se identifica el ángulo detectado para el vehículo.

Cando el sistema detecta que un vehículo pasa por el punto A, este guarda la información del estado del vehículo Figura \ref{fig:PuntoA}, puede haber múltiples vehículos pasando en ese momento y todos serán detectados por YOLO, sin embargo, para identificar el vehículo del cual se está guardando su estado con el recuadro blanco, ya que para el resto de vehículos se dibuja un recuadro amarillo.

\begin{figure}[H]
    \centering
    \includegraphics[width=0.8\textwidth]{Metodologia/imgs/Punto_A.jpg}
    \caption{Punto A con vehículo en recuadro blanco.}
    \label{fig:PuntoA}
\end{figure}

Aunque se guardó el estado del vehículo cuando paso por el punto A, este no genera una línea para el archivo CSV resultante, es hasta que el vehículo pasa por el punto B y coincide con los segundos en el archivo CSV de entrada que el sistema guarda un nuevo dato en el archivo de salida, los vehículos que no cuentan con una línea en el archivo CSV de entrada no se les guarda su información. La Figura \ref{fig:PuntoB} muestra cuando el vehículo detectado en el punto A pasa por el punto B.

\begin{figure}[H]
    \centering
    \includegraphics[width=0.8\textwidth]{Metodologia/imgs/Punto_B.jpg}
    \caption{Punto B con vehículo en recuadro blanco.}
    \label{fig:PuntoB}
\end{figure}

La Tabla \ref{tab:CaracteristicasSistema} muestra las características más importantes generadas por el sistema y su descripción.

\begin{table}[H]
    \caption{Características obtenidas por el sistema.}
    \label{tab:CaracteristicasSistema}
    \begin{tabular}{|l|l|}
        \hline
        \textbf{Característica} & \multicolumn{1}{c|}{\textbf{Descripción}} \\ \hline
        \textbf{Angulo Salida} & Angulo a partir del punto de entrada hasta el punto de salida \\ \hline
        \textbf{Distancia de salida} & Distancia recorrida desde el punto de entrada hasta el punto de salida \\ \hline
        \textbf{Área Entrada} & Área detectada del vehículo en pixeles en el punto de entrada \\ \hline
        \textbf{Área Salida} & Área detectada del vehículo en pixeles en el punto de salida \\ \hline
        \textbf{FPS} & Fotogramas por segundo del video \\ \hline
        \textbf{Tiempo} & Tiempo que le tomo al vehículo para pasar del punto entrada al de salida \\ \hline
        \textbf{Velocidad} & Velocidad detectada por el radar \\ \hline
        \textbf{Carril} & Carril por que pasa el vehículo \\ \hline
        \textbf{Identificador} & Identificador correspondiente a una imagen generada \\ \hline
    \end{tabular}
\end{table}


El sistema además de generar un archivo CSV, crea una imagen de salida la cual corresponde a una línea del archivo resultante, esta imagen está formada por dos imágenes una al lado de la otra, la imagen de la izquierda representa el vehículo cuando entra en el punto A y la imagen de la derecha es cuando el vehículo pasa por el punto B. La Figura \ref{fig:Completo} muestra un ejemplo de esta imagen de salida.

\begin{figure}[H]
    \centering
    \includegraphics[width=1\textwidth]{Metodologia/imgs/Completo.jpg}
    \caption{Imagen resultado para cada linea de archivo CSV.}
    \label{fig:Completo}
\end{figure}

%%%%%%%%%%%%%%%%%%%%%%%%%%%%%%%%%%%%%%%%%%%%%%%%%%%%%%%%%%%%%%%%%%%%%%%%%%%%%%%%
%%%%%%%%%%%%%%%%%%%%%%%%%%%%%%%%%%%%%%%%%%%%%%%%%%%%%%%%%%%%%%%%%%%%%%%%%%%%%%%%
%%%%%%%%%%%%%%%%%%%%%%%%%%%%%%%%%%%%%%%%%%%%%%%%%%%%%%%%%%%%%%%%%%%%%%%%%%%%%%%%
%%%%%%%%%%%%%%%%%%%%%%%%%%%%%%%%%%%%%%%%%%%%%%%%%%%%%%%%%%%%%%%%%%%%%%%%%%%%%%%%
%%%%%%%%%%%%%%%%%%%%%%%%%%%%%%%%%%%%%%%%%%%%%%%%%%%%%%%%%%%%%%%%%%%%%%%%%%%%%%%%

Una vez que el sistema genera nuestras imágenes de salida y el archivo CSV resultante se valida que el vehículo detectado sea el correcto y que el área de detección del vehículo sea lo más completa posible.

Para el caso de validar que el vehículo sea el correcto es necesario ver cada una de las imágenes generadas y leer la descripción en el archivo CSV de entrada, en caso de identificar un vehículo que no corresponda se debe modificar los segundos en el archivo CSV de entrada de tal manera que el sistema detecte el vehículo  correcto, existen dos casos en los que el sistema no podrá identificar el vehículo, que el vehículo correcto sea tapado por otro o que al vehículo no se le haya podido generar su seguimiento completo del punto A al B de forma correcta, para este caso se elimina la línea correspondiente en el archivo de entrada, para ambos casos ya sea que se tenga que modificar los segundos o se tenga que eliminar la línea es necesario ejecutar el sistema nuevamente para que el sistema genere todo de nuevo.

Para validar que el área de detección del vehículo sea lo más completa posible también se realiza una inspección visual de las imágenes generadas, en estas se identifica cuando un vehículo ha sido detectado completamente, como muestra la Figura \ref{fig:ImagenValida}.

\begin{figure}[H]
    \centering
    \includegraphics[width=1\textwidth]{Metodologia/imgs/Valido.jpg}
    \caption{Imagen valida representando una linea del archivo CSV.}
    \label{fig:ImagenValida}
\end{figure}

Por otra parte, hay ocasiones en la cuales el sistema solo detecta parte del vehículo, estas imágenes son las que podemos considerar como invalidas, un ejemplo se muestra en la Figura \ref{fig:ImagenInvalida}.

\begin{figure}[H]
    \centering
    \includegraphics[width=1\textwidth]{Metodologia/imgs/Invalido.jpg}
    \caption{Imagen invalida representando una linea del archivo CSV.}
    \label{fig:ImagenInvalida}
\end{figure}

Una vez que se identifica una imagen inválida se elimina la línea correspondiente en el archivo CSV de salida, esta línea se reconoce por el identificador del final, ya que este corresponde a la imagen. Para este caso no es necesario volver a ejecutar el sistema, pero se recomienda hacer un respaldo del archivo CSV original para futuras referencias.


\subsection{Modelo predictivo}
\label{cap:EstimacionVelocidad}

En esta sección se describe la metodología utilizada para crear un proceso o modelo el cual es utilizado para determinar la velocidad de vehículos. Se implementaron dos procesos para esta tarea. El primer proceso utiliza la fórmula física de la velocidad, a partir de esta fórmula se busca un factor de correlación entre la distancia en píxeles recorrida por el vehículo y la distancia en metros. Mientras que el segundo proceso implementado usa un modelo de aprendizaje máquina utilizando bibliotecas de Python las cuales ayudan a crear un modelo predictivo con el cual se determina la velocidad del vehículo.

\subsubsection{Estimar velocidad basado en factor de correlación}
\label{cap:EstimarVelocidadBasadoFactorCorrelacion}
Para determinar la velocidad utilizando el conjunto de datos generado por el sistema se inicia utilizando la fórmula física descrita en la Ecuación \ref{eq:Velocidad} para el cálculo de la velocidad.

\begin{equation}
    \label{eq:Velocidad}
    Velocidad = \frac{Distancia}{Tiempo}
\end{equation}

En este caso las muestras fueron tomadas por un Radar que solo proporciona la velocidad en Kilómetros o en Millas, así que necesitamos convertir este valor a metros sobre segundo, utilizando la Ecuación \ref{eq:ConvertMSKH}.

\begin{equation}
    \label{eq:ConvertMSKH}
    \frac{M}{S} = \frac{18}{5} \times \frac{K}{H}
\end{equation}

A partir de la Ecuación \ref{eq:Velocidad} donde se realiza cálculo de la velocidad y la conversión de la velocidad en metros por segundo en lugar de kilómetros por hora. La Figura \ref{fig:CrearModeloCustom} define los pasos que se deben seguir para modelar los datos obtenidos utilizando un factor de correlación.
Se define el proceso para crear el modelo en el diagrama de flujo que se muestra en la Figura \ref{fig:CrearModeloCustom}.

\begin{figure}[H]
    \centering
    \includegraphics[width=0.5\textwidth]{Metodologia/imgs/DFCrearModeloCustom.png}
    \caption{Diagrama de flujo para la creación de modelo con la fórmula física de la velocidad.}
    \label{fig:CrearModeloCustom}
\end{figure}

La Tabla \ref{tab:CaracteristicasSistema} muestra que los tres parámetros necesarios para estimar la velocidad, sin embargo, el valor de la distancia está en pixeles, por lo que es necesario convertir el valor a metros. Este valor es convertido con la Ecuación \ref{eq:Velocidad} despejando la distancia, obteniendo la  Ecuación \ref{eq:DistanciaEstimada}.

\begin{equation}
    \label{eq:DistanciaEstimada}
    Distancia\:Estimada = Tiempo \times Velocidad
\end{equation}

Una vez que se tiene la distancia estimada, se busca encontrar la correlación entre está y la distancia en pixeles, con lo cual se propone la Ecuación \ref{eq:EcuacionB}.

\begin{equation}
    \label{eq:EcuacionB}
    B = \frac{Distancia \: Estimada}{Distancia \: Pixeles}
\end{equation}

A la constante de correlación se le llamara B, Donde B es el valor utilizado para calcular una velocidad estimada con respecto a la distancia y el tiempo. Por lo que al ajustar la Ecuación \ref{eq:Velocidad} con la constante de correlación (Ecuación \ref{eq:EcuacionB}) tenemos la Ecuación \ref{eq:VelocidadB}.

\begin{equation}
    \label{eq:VelocidadB}
    Velocidad = \frac{Distancia \: Pixeles}{Tiempo} \times B
\end{equation}

Sin embargo, es necesario encontrar la correlación de B en todas las muestras. Por lo cual se propone encontrar un valor de B que modele la gran mayoría de muestras. Para esto se calculó el valor medio de B definido por la Ecuación \ref{eq:PromedioB}.

\begin{equation}
    \label{eq:PromedioB}
    \overline{B} = \frac{\sum B}{n}
\end{equation}

Una vez se tiene el promedio B ($\overline{B}$) se puede sustituir por B en la Ecuación \ref{eq:VelocidadB} lo cual quedaría como muestra la Ecuación \ref{eq:VelocidadBPromedio}.

\begin{equation}
    \label{eq:VelocidadBPromedio}
    Velocidad = \frac{Distancia \: Pixeles}{Tiempo} \times \overline{B}
\end{equation}



\subsubsection{Estimar velocidad utilizando modelos de regresión}
\label{cap:RegressionEstimar}

El proceso para inferir un dato (en este caso la velocidad) tanto para un modelo de regresión o de aprendizaje maquina es el mismo proceso. Este proceso inicia separando la información (en conjuntos de entrenamiento y validación), entrenando (refinando) el modelo predictivo, para finalmente estimar la velocidad (validar el modelo). La  Figura \label{ref:ModeloScikitTensorFlow} describe las tres etapas descritas para la obtencion de la velocidad.%que se requieren para entrenar un modelo.


\begin{figure}[H]
    \centering
    \includegraphics[width=0.5\textwidth]{Metodologia/imgs/DFDefinirModelo.png}
    \caption{Diagrama de flujo para definir un modelo predictivo de la velocidad.}
    \label{fig:ModeloScikitTensorFlow}
\end{figure}

El primer paso es separar el conjunto de datos en un conjunto de entrenamiento y validación. Normalmente esto se realiza en un porcentaje de 70\% (entrenamiento) - 30\% (validación). Una vez separado el conjunto de datos se procede a la optimización de los parámetros del modelo o también llamado entrenamiento. Por último, se utiliza el modelo con el conjunto de validación para validar que los resultados obtenidos.


\subsection{Obtención de la velocidad}

Utilizando el conjunto de datos creado por el sistema para generar el modelo que se propone en la Sección \ref{cap:EstimacionVelocidad} podemos determinar la velocidad simplemente adquiriendo los datos de la distancia en pixeles y el tiempo más nuestra B promedio, con esto es solo cuestión de sustituir los valores en la Ecuación \ref{eq:VelocidadBPromedio}.

Por ejemplo, calculamos nuestra B promedio la cual es igual a 0.0162, el sistema obtiene la distancia en pixeles y el tiempo que le tomo al vehículo pasar del punto A al punto B, los cuales son igual a 962 y 0.933 respectivamente, ahora solo es cuestión de sustituir estos valores para conseguir la velocidad estimada.

\begin{equation}
    \label{eq:EjemploImplementacion}
    Velocidad = \frac{962 }{0.933} \times 0.0162 = 16.703 \: m/s
\end{equation}


No obstante, esta estimación está en M/S, entonces se necesita convertirlo a K/H de la siguiente manera.

\begin{equation}
    \label{eq:EjmploKH}
     16.703\times \frac{18}{5} = 60.13 \: k/h
\end{equation}

Este es el resultado de estimar la velocidad de un vehículo utilizando el método propuesto, este es un método simple el cual ayudó como punto de referencia para futuros experimentos, por ejemplo, usando bibliotecas como Scikit para utilizar modelos ya existentes en esta biblioteca o Tensor Flow para crear una red neuronal con las características que mejor se adecuen a este caso.

