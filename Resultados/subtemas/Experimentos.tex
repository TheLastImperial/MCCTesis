\section{Experimentos}

Principalmente se implementan 3 experimentos, el primero es utilizando el cálculo mostrado en la Sección \ref{cap:EstimacionVelocidad}, el segundo es utilizando modelos implementados en la biblioteca Scikit Learn de Python y por último con el uso de Tensor Flow implementado una red neuronal de capas densas Figura \ref{fig:RedTF}.

En el caso de Scikit y Tensor Flow, se debe considerar que para cada entrenamiento se pueden obtener diferentes resultados. Esto debido a que los valores iniciales para cada modelo son aleatorios, entonces es posible tener diferentes precisiones entre un entrenamiento y otro, a pesar de que se utilice el mismo conjunto de datos. Por esta razón, el entrenamiento se realizó un número considerable de veces con el conjunto de datos creado, con lo cual se tienen diferentes resultados cada vez. Se presenta el valor máximo de error obtenido con respecto al total de veces que se repitieron los entrenamientos, el valor mínimo, el promedio y la desviación estándar de todos los entrenamientos.

Para todos los casos fue implementada la métrica de evaluación del Error Absoluto Medio, lo cual significa que las predicciones tienen en promedio un margen de error de N M/S o su equivalente en K/H, con esto proporciona una idea de que tan precisa es la predicción con los diferentes modelos implementados.

