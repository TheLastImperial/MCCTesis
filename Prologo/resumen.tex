
\noindent {\fontsize{24}{30}\selectfont \textbf{Resumen}}

\begin{spacing}{1.5}

La movilidad urbana se incrementó con el uso de automóviles, lo que originó también un aumento de accidentes de tránsito. Para estudiar este fenómeno se requiere de estudios de tránsito mediante equipo especial. Con los avances tecnológicos, la inteligencia artificial y el uso de videos es posible realizarlos sin modificar en gran medida la infraestructura urbana. Para el diseño de soluciones basadas en inteligencia artificial es necesario generar bases de datos públicas que proporcionen videos confiables para la calibración y desarrollo de soluciones que permitan realizar estudios de tránsito de manera automatizada. En este trabajo se presenta un sistema para generar un conjunto de datos a partir de videos grabados en un punto de observación de una vía carretera, obteniendo un total de 532 datos para analizar, los cuales fueron separados por el carril en el que fue detectado el vehículo. Este conjunto de datos fue utilizado para realizar experimentos con métodos estadísticos de correlación. La implementación de estos modelos de aprendizaje máquina se realizó mediante el uso bibliotecas como Scikit-Learn y Tensor Flow. Los mejores resultados arrojaron un Error Absoluto Medio (MAE) de 1.872 K/H para el carril central y 2.128 K/H para el último carril, aceptable comparados con el radar de velocidad Bushnell el cual tiene una precisión de 1.609 K/H. Debido a la falta de datos no fue posible realizar pruebas en el primer carril.  


%7.74 K/H y 5.86 K/H para el carril central y el último carril respectivamente utilizando métodos estadísticos, 1.246 K/H y 0.738 K/H para el carril central y el último carril utilizando la biblioteca Scikit y con 1.872 K/H y 2.128 K/H para el carril central y el último carril utilizando el modelo creado con Tensor Flow.


\end{spacing}