
\noindent {\fontsize{24}{30}\selectfont \textbf{Resumen}}

\begin{spacing}{1.5}

Este trabajo es parte del análisis de tráfico vehicular, Son muchos los puntos que se pueden estudiar sobre el tráfico vehicular, densidad vehicular, horas pico, accidentes, velocidad, etc. Sin embargo, este trabajo se enfoca en determinar la velocidad de vehículos, apoyándose en cámaras de video.

Existen dispositivos especializados para calcular la velocidad de vehículos y otros objetos, pero estos dispositivos se enfocan solo a esto, calcular velocidad. Agregar la capacidad de determinar la velocidad de vehículos a una cámara de video ayuda a estas cumplan doble funcionamiento, grabar video y proporcionar la velocidad.

Este trabajo integra inteligencia artificial para identificar vehículos y apoyarse para extraer las características más relevantes para crear un conjunto de datos con el cual podemos realizar experimentos y lograr la meta utilizando diferentes métodos, ya sean estadísticos o de aprendizaje máquina.

El sistema creado con este trabajo servirá como base para futuros trabajos relacionados al análisis de tráfico vehicular, agregando nueva funcionalidad o creando un conjunto de datos más basto de información importante que ayude mejorar las predicciones.

\end{spacing}