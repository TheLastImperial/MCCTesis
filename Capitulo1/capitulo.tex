\chapter{Introducción}

El aumento del uso de vehículos motorizados se debe al incremento de la población centrándose en las zonas urbanas y la necesidad de movilidad en la vida cotidiana. Como consecuencia se tiene un aumento de tráfico vehicular, lo cual conlleva a un mayor número de accidentes vehiculares e incremento de gases de efecto invernadero, afectando la salud de la población en general. El uso de automóviles es esencial en nuestra vida cotidiana. Según la organización \textit{Association for Safe International Road Travel (ASIRT)}, cada año mueren más de 1.3 millones de personas en accidentes de tráfico. Además,  entre 20 y 50 millones de personas resultan heridas o incapacitadas (\cite{zaki2020Traffic}).

En caso de accidente, la mayor responsabilidad recae en el conductor del automóvil (\cite{velazquez2017Siniestralidad}). Entre los principales factores que provocan los accidentes de tránsito se encuentran el exceso de velocidad, la conducción distraída, obstáculos en carretera, mala señalización, estado de la infraestructura vial, e iluminación. Según datos del Instituto Nacional de Estadística e Información Geográfica (INEGI, 2011), entre 1997 y 2009, los accidentes en la región aumentaron un 72.7\% en zonas urbanas y rurales (\cite{carro2019Conductas}).

Un elemento importante al revisar la causa de accidentes viales es el exceso de velocidad, por este motivo los estudios de tránsito vehicular se enfocan en revisar las causas de este elemento. Estos estudios requieren de un sistema monitoreo de velocidad eficaz. Además, se requiere de sistemas de control que asistan al conductor al transitar por las calles, conocidos como \textit{ADAS (Advanced Driver Assistance Systems)} (\cite{carro2019Conductas}).

Actualmente, para un estudio de tránsito se requiere de equipo especializado para generar los datos necesarios para analizar la situación y obtener posibles soluciones; uno de estos dispositivos son los radares de velocidad los cuales utilizan ondas de radio para calcular el tiempo que le toma a la onda viajar hasta el vehículo y de vuelta al radar para determinar la velocidad con la que viaja el vehículo. A su vez estos dispositivos son utilizados en puntos estratégicos de la ciudad para monitorear la velocidad de los vehículos y captar una imagen del vehículo para elaborar una multa por exceso de velocidad.

Ahora bien, para un estudio de tránsito se requieren equipos especializados como los radares de velocidad, pero usar estos equipos no es tan viable por su elevado costo. Sin embargo, actualmente las ciudades cuentan con un sistema de videovigilancia, en donde es posible usar este equipo existente en la ciudad para determinar varias características del flujo vehicular como la velocidad, accidentes, entre otros.

La estimación del tráfico puede proporcionarse a los usuarios finales, a los departamentos de seguridad del estado, a los departamentos de planeación, etc. entre otros (\cite{impedovo2019Vehicular}).

La inteligencia artificial está revolucionando las actividades cotidianas que lleva a cabo la sociedad moderna. Los investigadores y desarrolladores tanto de la industria automotriz como de seguridad vial están desarrollando activamente enfoques de conducción autónoma y monitoreo basados en el aprendizaje profundo. Para esto, es necesario contar con una red neuronal capaz de detectar vehículos capturados mediante video con la finalidad de determinar su velocidad, clasificarlos, detectar accidentes, por mencionar algunas tareas.

\cite{rao2018Deep} describen las posibilidades y desafíos de integrar el aprendizaje profundo en vehículos autónomos en donde se estudia la creación de base de datos para el entrenamiento de dichas redes neuronales. En este trabajo se presenta una herramienta para determinar la velocidad vehicular a partir de secuencias de imágenes. Para el desarrollo de esta herramienta fue necesario la elaboración de un conjunto de datos experimental propio para después con técnicas de inteligencia artificial y estadísticas extraer la información necesaria para la estimación de la velocidad.



\section{Definición del problema}

Actualmente hay un incremento de la población en las grandes ciudades, entre algunas razones del crecimiento es que, estas tienden a proporcionar más trabajos. Este incremento de población ha provocado diferentes problemas entre ellos el de movilidad. En las ciudades grandes o pequeñas, los vehículos de motor son el principal medio con el cual la población es capaz de trasladarse ya sea para la recreación o para llegar a sus lugares de trabajo. El aumento de población y aumento de vehículos también son causantes de gran cantidad de emisiones de gases de efectos invernaderos, los cuales a su vez provocan problemas de salud para la población en general.

Un flujo de vehículos sin pausas prolongadas ayudan a disminuir los problemas de tráfico y la reducción de contaminación. Mejorar el flujo vehicular es importante para disminuir las consecuencias de la gran cantidad de vehículos. El análisis de tráfico se utilizan para identificar características como las horas pico, accidentes viales, la velocidad, el conteo de vehículos, entre otros. El presente trabajo está centrado en el desarrollo de una herramienta capaz de determinar la velocidad a la que viajan los vehículos a partir de secuencias de imágenes.

El enfoque tradicional para realizar estudios de tránsito está influenciado por el uso de equipo especial. Estos se pueden instalar bajo la superficie de la carretera tales como espires inductivas, sensores de campo magnético, contadores de ejes, sensores capacitivos y piezoeléctricos. Por otro lado, se pueden instalar encima de la carretera, como los son los detectores de radares de microondas, detectores de radar láser, sensores de campo magnético, sensores infrarrojos pasivos y activos, sensores ultrasónicos, entre otros instrumentos; los cuales tienen un costo elevado además de necesitar una instalación especializada.

Las cámaras de videovigilancia son dispositivos instalados en puntos estratégicos de las grandes ciudades. Es por esto que se busca agregarles la capacidad de estimar la velocidad de los vehículos con la ayuda de inteligencia artificial. Buscando aprovechar la infraestructura existente y ahorrar en la adquisición de nuevos dispositivos especializados.

\section{Hipótesis}

Es posible determinar la velocidad de vehículos en movimiento a partir de secuencias de imágenes utilizando técnicas de inteligencia artificial y visión artificial.

\section{Objetivo}

Determinar la trayectoria y velocidad de un vehículo usando secuencias de imágenes a través de técnicas de inteligencia artificial y visión artificial.

\section{Objetivos específicos}

\begin{itemize}
    \item Generar un conjunto de datos que permitan diseñar y entrenar un modelo capaz de terminar la velocidad de un vehículo a partir de secuencias de imágenes.
    \item Diseñar e implementar un modelo que permita la detección de un vehículo. 
    \item Diseñar e implementar un algoritmo para determinar la trayectoria de un vehículo.
    \item Diseñar e implementar una herramienta para la determinación de la velocidad de vehículos mediante secuencias de imágenes.
\end{itemize}

\section{Justificación}

Con la implementación de un sistema para determinar la velocidad de vehículos en cámaras de videovigilancia, se proporciona a los centros de monitoreo datos para analizar e identificar los puntos más importantes en los que hay que poner especial atención para evitar posibles accidentes viales que podrían costar la pérdida de vidas o simplemente el retraso en los tiempos de traslados provocados por la congestión de tráfico lo cual a su vez ayuda a disminuir los gases de efecto invernadero nocivos para la salud.

Además, con la integración del sistema propuesto se espera un ahorro económico, puesto que no es necesario adquirir equipo especializado para determinar la velocidad de vehículos. Por ejemplo, de radar de velocidad Bushnell tiene un costo promedio de 6,000 pesos mexicanos. Sin embargo, este dispositivo solo puede guardar la velocidad en pantalla, sin guardar evidencia del vehículo al que se le tomó la velocidad. Por otra parte, con la ayuda de cámaras de videovigilancia y este sistema es posible agregar la funcionalidad para determinar la velocidad de vehículos, aprovechando equipos existentes en la mayoría de las grandes ciudades, además de darle doble funcionalidad a estos equipos a los cuales no hay que realizar una instalación extra que provoquen pausar el tráfico, como en el caso de lazos inductivos, ya que el sistema no es invasivo con sistemas de vigilancia existentes.


\section{Estructura de la tesis}

Este trabajo está dividido en seis capítulos organizados de la siguiente manera:


\begin{itemize}
\item Capítulo 1 - Introducción: Se define el problema, la hipótesis, el objetivo, los objetivos específicos y la justificación de este trabajo.

\item Capítulo 2 - Marco teórico: Se presentan los conceptos clave inteligencia artificial y visión artificial necesarios para comprender el contenido de este trabajo.

\item Capítulo 3 - Estado del arte: Se presentan trabajos relacionados con determinar la velocidad de objetos.

\item Capítulo 4 - Metodología: Se explica el proceso para el desarrollo del sistema, así como las herramientas utilizadas y la razón de utilizar ciertas herramientas.

\item Capítulo 5 - Análisis de los resultados: Se presentan los resultados obtenidos utilizando el sistema y comparándolo con métodos convencionales como los radares de velocidad.

\item Capítulo 6 - Conclusiones: Se presenta de manera resumida el proyecto, al mismo tiempo que el aporte del trabajo al área y el trabajo futuro que podría realizarse.
\end{itemize}
